\documentclass[lettersize,journal]{IEEEtran}
\usepackage{amsmath,amsfonts}
\usepackage{algorithmic}
\usepackage{algorithm}
\usepackage{array}
\usepackage[caption=false,font=normalsize,labelfont=sf,textfont=sf]{subfig}
\usepackage{textcomp}
\usepackage{stfloats}
\usepackage{url}
\usepackage{verbatim}
\usepackage{graphicx}
\usepackage{cite}
\hyphenation{op-tical net-works semi-conduc-tor IEEE-Xplore}
% updated with editorial comments 8/9/2021

\usepackage{booktabs} 
\usepackage{worldflags}
\flagsdefault[width=6pt, length=9pt, framewidth=0.1mm]

\begin{document}

\title{Analyzing social networks in the European Parliament, and changes in the social network over time}

\author{{BERNÁT Ádám, MARITS Márton}
        % <-this % stops a space
%\thanks{This paper was produced by us.}% <-this % stops a space
%\thanks{Manuscript received June 20, 2023; published two seconds later.}
}

% The paper headers
\markboth{Journal of BME Témalabor,~Vol.~1, No.~1, June~2023}%
{Shell \MakeLowercase{\textit{et al.}}: A Sample Article Using IEEEtran.cls for IEEE Journals}

%\IEEEpubid{0000--0000/00\$00.00~\copyright~2021 IEEE}
% Remember, if you use this you must call \IEEEpubidadjcol in the second
% column for its text to clear the IEEEpubid mark.

\maketitle

\begin{abstract}
We analyze a dataset of amendment co-sponsorship in the European Parliament. From this dataset, we construct a graph representing the co-sponsorship social network of the European Parliament, and we analyze its changes over time, paying special attention to the effects of major events, such as Brexit, COVID-19 and the Russo-Ukrainian war. We also analyze the breakdown of the social network by country and by political group.
\end{abstract}

\begin{IEEEkeywords}
Social networks, political groups, European Parliament, graph projections.
\end{IEEEkeywords}

\section{Introduction}

%THIS IS THE LAST SECTION WE WILL COMPLETE

%In this section, we introduce what the hell we're even talking about, we cite a couple earlier papers which we definitely read and got inspired from, we talk about previous results and so on.

The European Parliament (EP for short) is a legislative institution of the European Union, in which representatives from each of the 28 (27 after Brexit) member states vote on legistation concerning the European Union. The Parliament consists of \textit{Members of the European Parliament} (MEPs), each of whom have a well-defined country of origin and political party. The political parties of each MEP are specific to their country of origin, but parties holding similar views organize themselves into \textit{political groups}, which act as super-parties in the context of the European Parliament.

% ['EPP', 'ECR', 'ID', 'Greens/EFA', 'RE', 'S&D', 'NI', 'GUE/NGL', '']
The major political groups in the EP are the European People's Party (EPP, centre-right), the Progressive Alliance of Socialists and Democrats (S\&D, left), Renew Europe (RE, liberal), the Greens–European Free Alliance (Greens/EFA, green), European Conservatives and Reformists (ECR, right), Identity and Democracy (ID, far-right) and The Left in the European Parliament (GUE/NGL, left). Representatives who do not belong to any of these groups are usually called Non-iscrits (French for `not registered'), often abbreviated as NI.

Our aim is to analyze the social networks of the European Parliament, and the temporal changes incited by major global events, such as the global COVID-19 pandemic, and the Russo-Ukrainian war.

Some analysis of social networks in the European Union has already been done in \cite{Baller} and \cite{Cherepnalkoski}. Work that is analogous to ours in the context of the United States Congress and Senate has also been seen before in \cite{Desmarais}, \cite{Fowler}, \cite{Porter} and \cite{Tanger}; while in \cite{Fischer}, Fischer et al. conducted an analysis of such co-sponsorship networks in the Swiss parliament.

%We then talk about how the paper itself is organized, and very lengthily talk about whan kinds of stuff is contained in each section. This is of course for the reader's convenience, and definitely not an attempt to make this paper longer than it should be.

The present paper is organized as follows. In Section \ref{sec:data}, we discuss our dataset, and how we prepared our data for analysis. We also briefly mention the limitations of the dataset. In Section \ref{sec:prelim_anal}, we analyze the dataset in its entirety, without regard for the changes over time. We do this to familiarize the reader with the dataset, and to thus allow a deeper understanding of the European Parliament, before conducting the main part of our analysis, in Section \ref{sec:overtime_anal}. In Section \ref{sec:overtime_anal}, we dissect our dataset into a series of discrete time intervals, and we will analyze the evolution of the dataset over time. We will take special measures to make sure that the effects of certain major events are pronounced in our analysis. We do this by partitioning our data into meaningful intervals, and conducting our analysis by analyzing how the social network changed between these intervals. Finally, in Section \ref{sec:conclusion} we conclude our analysis by giving a few final remarks, and in Section \ref{sec:futureresearch}, we discuss future research plans, and other possible research topics related to this one.

\section{Our data} \label{sec:data}

Our dataset was acquired directly from the European Parliament's website. It is organized as a \texttt{csv} file which contains entries for each proposed amendment to a law, with information about when the amendment was proposed, some details about the amendment, and more importantly, information about who proposed the amendment, what party they belong to, which EP group said party belongs to, and which country are they are representative of.


This data can therefore be viewed as a bipartite graph, in which one part consists of the MEPs, and the other consists of the proposed amendments. An MEP and an amendment are joined by an edge if and only if said MEP contributed to the amendment (\textit{sponsored} the amendment). Importantly, a single amendment may have multiple contributors, which allows us to analyze the social structure of the European Parliament as a whole.

In total, our dataset contains 750,578 entries, which is the total number of edges in this bipartite graph. The dataset has data on a grand total of 754 MEPs. %% lehetne ebből valami breakdown, hogy hány MEP van országonként, pártonként, pártcsoportonként stb.

To analyze the social structure, we \textit{projected} this bipartite graph onto the set of MEPs. This procedure creates a new graph, wherein the nodes represent MEPs, and each edge connects two MEPs which have a contribution in common. (I.e. they \textit{co-sponsored} a bill.)

With this projection, a new, not necessarily bipartite graph was created, which represents the state of the social network of the European Parliament. Notably, the number of contributions in common are not considered. An edge between two MEPs in this graph might represent a single amendment they co-sponsored, or it might represent a long-lasting relationship of co-sponsorship. The projection unfortunately obscures the details of the strength of the co-sponsorship based connection.

To account for this, we also worked with a different type of projection, a \textit{weighted projection}. This modification of the projection algorithm serves to solve the problem outlined above. With a weighted projection, as previously, we create a new graph, whose nodes represent the MEPs, and connect to MEPs if they have co-sponsored a bill together. However, we also account for the number of co-contributions, which become the weight of each edge.

%% In the future, we're also planning to work with \textit{weighted projections}, which will allow for a more granular analysis of the social network.

\section{Preliminary analysis} \label{sec:prelim_anal}

In this section, we talk about the conclusions we can draw from the data while analyzing it in its entirety. This phase of our analysis allowed us to get a sense of what the data looks like from a bird's eye view.

\subsection{Analysis of activity by country} \label{countryactivity}

% {'Czechia': 1020, 'Poland': 2531, 'France': 4243, 'Netherlands': 1941, 'Slovakia': 1201, 'Romania': 2497, 'Spain': 3853, 'Croatia': 636, 'Greece': 1092, 'Portugal': 1406, 'Italy': 3550, 'United Kingdom': 268, 'Belgium': 1411, 'Germany': 5539, 'Finland': 859, 'Bulgaria': 1260, 'Ireland': 806, 'Hungary': 1135, 'Sweden': 1613, 'Malta': 614, 'Austria': 1272, 'Slovenia': 760, 'Latvia': 426, 'Cyprus': 502, 'Luxembourg': 623, 'Lithuania': 541, 'Estonia': 423, 'Denmark': 955, '': 175}

We conducted an analysis of how many amendments were produced in each country. The data we acquired can be seen in Figure \ref{amendments_by_country}. Besides the contributions visible in the table, our data also included 175 data points, which did not have an MEP attached. For the remainder of this analysis, we will discard these data points.

We direct the reader's attention to the fact that representatives of the United Kingdom were the least active in this period. This is unsurprising, since these representatives left the European Parliament early, in February 2020.

\begin{figure}[h]
	\begin{center}
		\begin{tabular}{| l | r |}
			\hline
			Country & Contributions  \\
			\hline
			\worldflag{DE} Germany & 5539 \\
			\worldflag{FR} France & 4242 \\
			\worldflag{ES} Spain & 3853 \\
			\worldflag{IT} Italy & 3550 \\
			\worldflag{PL} Poland & 2531 \\
			\worldflag{RO} Romania & 2497 \\
			\worldflag{NL} Netherlands & 1941 \\
			\worldflag{SE} Sweden & 1613 \\
			\worldflag{BE} Belgium & 1411 \\
			\worldflag{PT} Portugal & 1406 \\
			\worldflag{AT} Austria & 1272 \\
			\worldflag{BG} Bulgaria & 1260 \\
			\worldflag{SK} Slovakia & 1201 \\
			\worldflag{HU} Hungary & 1135 \\
			\worldflag{GR} Greece & 1092 \\
			\worldflag{CZ} Czechia & 1020 \\
			\worldflag{DK} Denmark & 955 \\
			\worldflag{FI} Finland & 859 \\
			\worldflag{IE} Ireland & 806 \\
			\worldflag{SI} Slovenia & 760 \\
			\worldflag{HR} Croatia & 636 \\
			\worldflag{LU} Luxembourg & 623 \\
			\worldflag{MT} Malta & 614 \\
			\worldflag{LT} Lithuania & 541 \\
			\worldflag{CY} Cyprus & 502 \\
			\worldflag{LV} Latvia & 426 \\
			\worldflag{EE} Estonia & 423 \\
			\worldflag{GB} United Kingdom & 268 \\
			\hline
		\end{tabular}
		\caption{Amendments by country}
		\label{amendments_by_country}
	\end{center}
\end{figure}

Of course, this data is heavily influenced by the uneven population distribution within the Union. We can see that the most populous countries are among the most active as well.


This is also unsurprising, however, in view of the fact that smaller countries get disproportionately more representatives per capita. While each member state of the Union gets a certain number of representatives based on its population, the process through which representatives are chosen is not that clear cut. A set of rules are in place to ensure that smaller countries get fairer representation, letting smaller countries have more MEPs for each unit of population. Because of this, Figure \ref{amendments_by_country_per_capita} is skewed towards smaller states.

%TODO:  MAKE ANOTHER CHART THAT IS NORMALIZED FOR NUMBER OF REPRESENTATIVES

%{'Czechia': 48.57142857142857, 'Poland': 49.627450980392155, 'France': 57.33783783783784, 'Netherlands': 74.65384615384616, 'Slovakia': 92.38461538461539, 'Romania': 78.03125, 'Spain': 66.43103448275862, 'Croatia': 57.81818181818182, 'Greece': 52.0, 'Portugal': 66.95238095238095, 'Italy': 48.63013698630137, 'United Kingdom': 3.671232876712329, 'Belgium': 67.19047619047619, 'Germany': 57.697916666666664, 'Finland': 66.07692307692308, 'Bulgaria': 74.11764705882354, 'Ireland': 73.27272727272727, 'Hungary': 54.04761904761905, 'Sweden': 80.65, 'Malta': 102.33333333333333, 'Austria': 70.66666666666667, 'Slovenia': 95.0, 'Latvia': 53.25, 'Cyprus': 83.66666666666667, 'Luxembourg': 103.83333333333333, 'Lithuania': 49.18181818181818, 'Estonia': 70.5, 'Denmark': 73.46153846153847}


\begin{figure}[h]
	\begin{center}
		\begin{tabular}{| l | l | r |}
			\hline
			Country & \# MEPs & Contributions per MEP  \\
			\hline
			\worldflag{LU} Luxembourg & 6 & 103.83 \\
			\worldflag{MT} Malta & 6 & 102.33 \\
			\worldflag{SI} Slovenia & 8 & 95.00 \\
			\worldflag{SK} Slovakia & 13 & 92.38 \\
			\worldflag{CY} Cyprus & 6 & 83.67 \\
			\worldflag{SE} Sweden & 20 & 80.65 \\
			\worldflag{RO} Romania & 32 & 78.03 \\
			\worldflag{NL} Netherlands & 26 & 74.65 \\
			\worldflag{BG} Bulgaria & 17 & 74.12 \\
			\worldflag{DK} Denmark & 13 & 73.46 \\
			\worldflag{IE} Ireland & 11 & 73.27 \\
			\worldflag{AT} Austria & 18 & 70.67 \\
			\worldflag{EE} Estonia & 6 & 70.50 \\
			\worldflag{BE} Belgium & 21 & 67.19 \\
			\worldflag{PT} Portugal & 21 & 66.95 \\
			\worldflag{ES} Spain & 58 & 66.43 \\
			\worldflag{FI} Finland & 13 & 66.08 \\
			\worldflag{HR} Croatia & 11 & 57.82 \\
			\worldflag{DE} Germany & 96 & 57.70 \\
			\worldflag{FR} France & 74 & 57.34 \\
			\worldflag{HU} Hungary & 21 & 54.05 \\
			\worldflag{LV} Latvia & 8 & 53.25 \\
			\worldflag{GR} Greece & 21 & 52.00 \\
			\worldflag{PL} Poland & 51 & 49.63 \\
			\worldflag{LT} Lithuania & 11 & 49.18 \\
			\worldflag{IT} Italy & 73 & 48.63 \\
			\worldflag{CZ} Czechia & 21 & 48.57 \\
			\worldflag{GB} United Kingdom & 73 & 3.67 \\
			\hline
		\end{tabular}
		\caption{Amendments by country normalized for number of MEPs (pre-Brexit)}
		\label{amendments_by_country_per_MEP}
	\end{center}
\end{figure}

To account for this, we have created another normalized chart, which normalizes the data for the number of representatives each country has. On Figure \ref{amendments_by_country_per_MEP} we can see this data compiled into a table.

This data still skews heavily towards smaller countries having more active representatives, and larger countries having lazier ones. Notably, the top two countries on this table, Luxembourg and Malta, are also the two least populous.

\subsection{Analysis of activity by political group} \label{epgroupactivity}

Let us now turn to analyzing the political groups of the European Parliament. The political groups are overarching organizations that parties from any country can join, allowing them to have a unified voice in the European Parliament. In this subsection, we will take a look at how active each political group is. We investigate how many contributions were made by each political group, in total and by number of representatives per political group.

% {'EPP': 13384, 'ECR': 2118, 'ID': 2090, 'Greens/EFA': 2288, 'RE': 8995, 'S&D': 11672, 'NI': 614, 'GUE/NGL': 1816}

\begin{figure}[h]
	\begin{center}
		\begin{tabular}{| l | r |}
			\hline
			EP Group & Contributions\\
			\hline
			EPP & 13384 \\
			S\&D & 11672 \\
			RE & 8995 \\
			Greens/EFA & 2288 \\
			ECR & 2118 \\
			ID & 2090 \\
			GUE/NGL & 1816 \\
			NI & 614 \\
			\hline
		\end{tabular}
		\caption{Contributions to the European Parliament by political group}
		\label{epgroup_contributions}
	\end{center}
\end{figure}

% {'EPP': 73.53846153846153, 'ECR': 34.16129032258065, 'ID': 28.63013698630137, 'Greens/EFA': 30.91891891891892, 'RE': 83.28703703703704, 'S&D': 75.79220779220779, 'NI': 11.37037037037037, 'GUE/NGL': 44.292682926829265}
\begin{figure}[h]
	\begin{center}
		\begin{tabular}{| l | r | r |}
			\hline
			EP Group & \#MEPs & Contribs/MEP\\
			\hline
			RE & 108 & 83.29 \\
			S\&D & 154 & 75.79 \\
			EPP & 182 & 73.54 \\
			GUE/NGL & 41 & 44.29 \\
			ECR & 62 & 34.16 \\
			Greens/EFA & 74 & 30.92 \\
			ID & 73 & 28.63 \\
			NI & 54 & 11.37 \\
			\hline
		\end{tabular}
		\caption{Contributions to the European Parliament by political group, normalized by number of MEPs}
		\label{epgroup_contributions_per_mep}
	\end{center}
\end{figure}

On Figures \ref{epgroup_contributions} and \ref{epgroup_contributions_per_mep} we tabulated our data on each political group. On Figure \ref{epgroup_contributions}, we see the number of contributions each EP group made to the proceedings of the European Parliament, while on Figure \ref{epgroup_contributions_per_mep}, we see this data normalized based on the number of MEPs belonging to each EP group.

What we can see from the normalized data is that in terms of contributions, the left-wing and liberal parties RE and S\&D do the most in the European Parliament, and similarly, among the smaller parties, we see left-wing GUE/NGL punching above its weight.

At the same time, we see the far-right party ID not contributing nearly as much to amendments in the European Parliament. We believe that this has to do with ID's continued euroskeptic ideology. The other EP group that is trailing the chart, the NI, also includes many euroskeptics. A significant portion of right-wing nationalist parties are contained in this group.

As such, we may conclude that according to our data, left-wing political groups do more work in the European Parliament.

\subsection{The MEPs with the highest degree}

As we have turned our data into a graph of the social structure of the European Parliament, with nodes for each MEP, it makes sense to find out which MEPs have the highest degree in this graph. In other words, we want to find out which MEP co-sponsored the most bills.

\begin{figure}[h]
	\begin{center}
		\begin{tabular}{| l | l | r |}
			\hline
			Name & Pol. group & Degree  \\
			\hline
			\worldflag{BE} Olivier Chastel & RE & 5155 \\
			\worldflag{ES} Lina Gálvez Muñoz & S\&D & 4483 \\
			\worldflag{LU} Marc Angel & S\&D & 4350 \\
			\worldflag{PT} Maria-Manuel Leitão-Marques & S\&D & 4245 \\
			\worldflag{RO} Maria Grapini & S\&D & 4200 \\
			\worldflag{RO} Nicolae Ștefănuță & RE & 3914 \\
			\worldflag{SI} Milan Brglez & S\&D & 3650 \\
			\worldflag{RO} Ramona Strugariu & RE & 3544 \\
			\worldflag{PT} Manuel Pizarro & S\&D & 3533 \\
			\worldflag{RO} Dragoș Pîslaru & RE & 3491 \\
			\hline
		\end{tabular}
		\caption{Top 10 MEPs by degree in the MEP to amendment bipartite graph}
		\label{top10_meps_degree}
	\end{center}
\end{figure}

On Figure \ref{top10_meps_degree} we can see the top 10 MEPs by degree in this social network graph. We can observe that only members of S\&D and RE can be found in the top ten. Seemingly, only the members of these two liberal, left-wing political groups go above and beyond in terms of productivity in the European Parliament. This is consistent with the findings of Subsection \ref{epgroupactivity}, where we concluded that on average, left-wing parties contribute to more amendments. This additional data suggest that not only do they contribute more on average, they also house the most productive members of the European Parliament.

Notably, four of the ten most active MEPs are from Romania. Romania and Portugal seem overrepresented among the most productive MEPs, along with other mid-sized and smaller countries. It is to be noted that in the top ten, none of the top three most populous countries, Germany, France and Italy, appear; despite the large number of representatives they send to the European Union. We can compare this to the data in Subsection \ref{countryactivity}, where we can see that in general, MEPs from more populous countries contribute less to the European Parliament.

\subsection{Degree centrality in the social network}

We have thus found the MEPs with the highest degree in the original, bipartite graph. Let us now find the MEPs with the highest degree in the projected social network graph. The difference from the previous subsection is that we are only considering how many co-sponsorship partners a certain MEP has, disregarding the amount of bill they co-sponsored.

\begin{figure}[h]
	\begin{center}
		\begin{tabular}{| l | l | r |}
			\hline
			Name & Pol. group & Degree  \\
			\hline
			\worldflag{BE} Hilde Vautmans & RE & 183 \\
			\worldflag{DK} Karen Melchior & RE & 173 \\
			\worldflag{LU} Marc Angel & S\&D & 169 \\
			\worldflag{BE} Olivier Chastel & RE & 167 \\
			\worldflag{PL} Łukasz Kohut & S\&D & 166 \\
			\worldflag{SK} Michal Šimečka & RE & 160 \\
			\worldflag{SK} Michal Wiezik & EPP & 152 \\
			\worldflag{RO} Ramona Strugariu & RE & 151 \\
			\worldflag{LT} Petras Auštrevičius & RE & 151 \\
			\worldflag{IE} Maria Walsh & EPP & 146 \\
			\hline
		\end{tabular}
		\caption{Top 10 MEPs by degree in the social network}
		\label{top10_meps_degreecentrality}
	\end{center}
\end{figure}

On Figure \ref{top10_meps_degreecentrality}, we can see the top 10 MEPs by degree centrality, i.e. by highest degree in the social network. We can see that some MEPs from the previous table also made the top 10 in this different measure, although there are also new faces.

We still see the dominance to the EP groups RE and S\&D, although in this chart, a lone member of EPP can also be found. Once again, we mostly see people from minor and mid-sized countries. The largest countries, such Italy, France and Germany have not produced any well-connected MEPs by this measure either. We see countries like Belgium, Romania and Slovakia overrepresented.


\subsection{Consistent contributors}

A natural question that might arise when researching this topic, is how many MEPs even managed to work throughout the entire 3-year period covered by our data. That is, how many MEPs contributed to at least \textit{some} amendments across every interval. We found that 195 MEPs of the 705 contributed to an amendment at least once in every time period. This constitutes 27.7\% of MEPs.

We then broke this data down by country and by political group. We found that all 27 member states (except for the UK) and all 8 political groups were present in the breakdown, although not evenly. Among countries, we found Italy to have the most MEPs that contributed in every interval, 27, while Germany, despite being the country with the most representatives in the European Parliament by far, only had 16 such contributors. The full data can be found in Figure \ref{consistent_contributors_by_country}.

Among political groups, we found that the general result of Section \ref{sec:prelim_anal}, that is, left-wing parties contributing more, still appears to be true. We found that the two parties with the highest number of consistent contributors were RE with 55, and S\&D with 42, both overtaking the largest political group, EPP, which only had 34 consistently productive MEPs. The full data broken down by political group can be found in Figure \ref{epgroup_consistent_contributors}.

%{'Spain': 20, 'Romania': 10, 'Belgium': 6, 'Italy': 29, 'Malta': 3, 'Germany': 16, 'Slovakia': 7, 'Ireland': 4, 'Slovenia': 3, 'Poland': 8, 'France': 25, 'Croatia': 1, 'Czechia': 8, 'Finland': 3, 'Netherlands': 11, 'Luxembourg': 3, 'Lithuania': 4, 'Austria': 3, 'Sweden': 7, 'Bulgaria': 5, 'Greece': 7, 'Portugal': 5, 'Hungary': 2, 'Estonia': 2, 'Latvia': 1, 'Denmark': 1, 'Cyprus': 1}
% egyenlőségnél angol név szerinti NÉVSOR SZERINT rendezem

\begin{figure}[h]
	\begin{center}
		\begin{tabular}{| l | r |}
			\hline
			Country & Consistent MEPs  \\
			\hline
			\worldflag{IT} Italy & 29 \\
			\worldflag{FR} France & 25 \\
			\worldflag{ES} Spain & 20 \\
			\worldflag{DE} Germany & 16 \\
			\worldflag{NL} Netherlands & 11 \\
			\worldflag{RO} Romania & 10 \\
			\worldflag{CZ} Czechia & 8 \\
			\worldflag{PL} Poland & 8 \\
			\worldflag{GR} Greece & 7 \\
			\worldflag{SK} Slovakia & 7 \\
			\worldflag{SE} Sweden & 7 \\
			\worldflag{BE} Belgium & 6 \\
			\worldflag{BG} Bulgaria & 5 \\
			\worldflag{PT} Portugal & 5 \\
			\worldflag{IE} Ireland & 4 \\
			\worldflag{LT} Lithuania & 4 \\
			\worldflag{AT} Austria & 3 \\
			\worldflag{FI} Finland & 3 \\
			\worldflag{LU} Luxemburg & 3 \\
			\worldflag{MT} Malta & 3 \\
			\worldflag{SI} Slovenia & 3 \\
			\worldflag{EE} Estonia & 2 \\
			\worldflag{HU} Hungary & 2 \\
			\worldflag{HR} Croatia & 1 \\
			\worldflag{CY} Cyprus & 1 \\
			\worldflag{DK} Denmark & 1 \\
			\worldflag{LV} Latvia & 1 \\	
			\hline
		\end{tabular}
		\caption{Consistent contributors by country}
		\label{consistent_contributors_by_country}
	\end{center}
\end{figure}

\begin{figure}[h]
	\begin{center}
		\begin{tabular}{| l | r |}
			\hline
			EP Group & Contributions\\
			\hline
			RE & 55 \\
			S\&D & 42 \\
			EPP & 34 \\	
			ECR & 20 \\
			ID & 19 \\
			GUE/NGL & 13 \\
			Greens/EFA & 8 \\
			NI & 4 \\
			\hline
		\end{tabular}
		\caption{Consistent contributors by EP group}
		\label{epgroup_consistent_contributors}
	\end{center}
\end{figure}

%\subsection{Contributions over time}

%???


\section{Conclusion} \label{sec:conclusion}

%In conclusion, we can say that the social networks of the European Parliament are an ever-changing structure. Some changes to it happen seemingly spontaneously, while some of them happen as a result of important events. The events that can have a noticeable effect on this social structure can be internal (such as Fidesz leaving the EPP group), or external, such as COVID-19, or the Russo-Ukrainian war.

\section{Future reserach} \label{sec:futureresearch}

%In the future, we plan to expand our current work by collecting newer data, and analyzing the development of the European Parliament social network in the current day. We are especially interested in how the social network changes with respect to the major events of the Russo-Ukrainian war. Unfortunately, due to the limitations of the data, we were only able to study the effects of the start of the war, as we only had data up to September 2022. Since many major events in the Russo-Ukrainian war happened since September 2022, we are more than interested in studying their effects on the European Union, specifically the European Parliament.

%Furthermore, we plan to apply even more mathematical methods to the dataset. %TODO: elaborate

\section*{Acknowledgments}

Thanks so much to everyone involved in creating this wonderful masterpiece of a paper.

\begin{thebibliography}{1}
\bibliographystyle{IEEEtran}

\bibitem{Baller}
Baller, Inger. "Specialists, party members, or national representatives: Patterns in co-sponsorship of amendments in the European Parliament." European Union Politics 18.3 (2017): 469-490.

\bibitem{Cherepnalkoski}
Cherepnalkoski, Darko, and Igor Mozetič. "Retweet networks of the European Parliament: Evaluation of the community structure." Applied network science 1 (2016): 1-20.

%\bibitem{citypopulation}
%\url{https://www.citypopulation.de/}

\bibitem{Desmarais}
Desmarais, Bruce A., et al. "Measuring legislative collaboration: The Senate press events network." Social Networks 40 (2015): 43-54.

\bibitem{Fischer}
Fischer, Manuel, et al. "How MPs ties to interest groups matter for legislative co-sponsorship." Social networks 57 (2019): 34-42.

\bibitem{Fowler}
Fowler, James H. "Connecting the Congress: A study of cosponsorship networks." Political analysis 14.4 (2006): 456-487.

\bibitem{McPherson}
McPherson, Miller, Lynn Smith-Lovin, and James M. Cook. "Birds of a feather: Homophily in social networks." Annual review of sociology 27.1 (2001): 415-444.

\bibitem{Neal}
Neal, Zachary. "The backbone of bipartite projections: Inferring relationships from co-authorship, co-sponsorship, co-attendance and other co-behaviors." Social Networks 39 (2014): 84-97.

\bibitem{Peixoto}
Peixoto, Tiago P., and Martin Rosvall. "Modelling sequences and temporal networks with dynamic community structures." Nature communications 8.1 (2017): 582.

\bibitem{Porter}
Porter, Mason A., et al. "A network analysis of committees in the US House of Representatives." Proceedings of the National Academy of Sciences 102.20 (2005): 7057-7062.

\bibitem{Tanger}
Tanger, Shaun M., and David N. Laband. "An empirical analysis of bill co-sponsorship in the US Senate: The Tree Act of 2007." Forest Policy and Economics 11.4 (2009): 260-265.

\end{thebibliography}


\vfill

\end{document}


